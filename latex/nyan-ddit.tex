\documentclass[labelsBySect]{seto}
\usepackage{relsize}
\title{Projective geo\\[0.2em]\smaller{}(and popular culture?)}
\displaytitle{not ddit}
\author{Neal Yan}\date\today
\setcounter{tocdepth}1
\begin{document}
\maketitle
\quote{If I were to take an all-geometry USAMO, I'd make Black MOP and you (two) would too...}{Tiger Zhang}
We assume a basic knowledge of harmonics and homography. (refer to, say, chapter 9 of )
\toc
\section{Projective warmups}
Some familiar examples, the middle three of which are from AIME. (Some are trivialized by a single perspectivity.)
\exercise[AoPS Olympiad Geometry 12/1???] Let circle $\omega$ touch line $\ell$ at a point $B$, which has antipode $A$ on that circle. Let $P$ be an arbitrary point outside $\omega$, and let the tangents from $P$ to $\omega$ touch it at $X,Y$. Then, $\ol{AP}\cap\ell$ is the midpoint of the segment with endpoints $\ol{AX}\cap\ell,\ol{AY}\cap\ell$.	
\exercise[2016 II/10] Triangle $ABC$ is inscribed in circle $\omega$. Points $P$ and $Q$ are on side $\overline{AB}$ with $AP<AQ$. Rays $CP$ and $CQ$ meet $\omega$ again at $S$ and $T$ (other than $C$), respectively. If $AP=4,PQ=3,QB=6,BT=5,$ and $AS=7$, then what is $ST$?
\exercise[2018 II/14]The incircle $\omega$ of triangle $ABC$ is tangent to $\overline{BC}$ at $X$. Let $Y \neq X$ be the other intersection of $\overline{AX}$ with $\omega$. Points $P$ and $Q$ lie on $\overline{AB}$ and $\overline{AC}$, respectively, so that $\overline{PQ}$ is tangent to $\omega$ at $Y$. Assume that $AP = 3$, $PB = 4$, $AC = 8$. What is $AQ$?
\exercise[2019 I/15]Let $\overline{AB}$ be a chord of a circle $\omega$, and let $P$ be a point on the chord $\overline{AB}$. Circle $\omega_1$ passes through $A$ and $P$ and is internally tangent to $\omega$. Circle $\omega_2$ passes through $B$ and $P$ and is internally tangent to $\omega$. Circles $\omega_1$ and $\omega_2$ intersect at points $P$ and $Q$. Line $PQ$ intersects $\omega$ at $X$ and $Y$. Assume that $AP=5$, $PB=3$, $XY=11$. What is $PQ^2$?
\section{Some other projective favorites}
Harmonics fun.
\exercise[Shortlist 2015 G3, due to Vivian Loh]Let $ABC$ be a triangle with $\angle C = 90^{\circ}$, and let $H$ be the foot of the altitude from $C$. A point $D$ is chosen inside the triangle $CBH$ so that $\overline{CH}$ bisects $\overline{AD}$. Let $P$ be the intersection point of the lines $\overline{BD}$ and $\overline{CH}$. Let $\omega$ be the semicircle with diameter $\overline{BD}$ that meets the segment $CB$ at an interior point. A line through $P$ is tangent to $\omega$ at $Q$. Prove that the lines $\overline{CQ}$ and $\ol{AD}$ meet on $\omega$.
\exercise[China Southeast 2018/5]In the isosceles triangle $ABC$ with $AB=AC$, the center of a circle $\omega$ is the midpoint of the side $BC$, and $\ol{AB},\ol{AC}$ are tangent to the circle at points $E,F$ respectively. Point $G$ is on $\omega$ with $\angle AGE = 90\deg$. A tangent line of $\omega$ passes through $G$, and meets $\ol{AC}$ at $K$. Prove that line $BK$ bisects $\ol{EF}$.

\section{Setup}
At the end we'll get to ``one of the latest fads in olympiad geometry'' (Evan Chen, OTIS homography)\dots\\[4pt]
Recall the following:
\begin{thm}[Definition \mdseries(involution)]
An involution (or involutive pairing) on a line is a map that is its own inverse, and preserves cross-ratio.
\end{thm}
Thus, we can project involutions on a line to those on a pencil of lines through a point, and vice versa.
\begin{thm}[Theorem \mdseries(characterization)]
All non-identical involutions on a line are inversions (possibly with negative power) about some point, possibly at infinity (in which case the inversion is a simple reflection about some point on the line.)
\end{thm}
\begin{proof}
Routine.
\end{proof}
\section{Desargues involution}
\subsection{The original}
\begin{thm}[Theorem \mdseries(Desargues's involution theorem aka DIT)]
Let quadrilateral $ABCD$ have circumconic $\gamma$, and let $\ell$ be any line. Then, there is an involution on $\ell$ swapping $(\ol{AB}\cap\ell,\ol{CD}\cap\ell)$, $(\ol{AD}\cap\ell,\ol{BC}\cap\ell)$, $(\ol{AC}\cap\ell,\ol{BD}\cap\ell)$, and the two points $\gamma\cap\ell$.
\end{thm}
I think this is proved using homography, but none of the people who taught me it bothered to prove it. Perhaps it isn't that enlightening.
From my olympiad experience I'd say this theorem itseif isn't that useful.
\exercise[Butterfly theorem] In circle $\omega$, $M$ lies on the chords $\ol{AC},\ol{BD}$, $\ol{PQ}$, and is the midpoint of the last. Then $M$ is equidistant from $\ol{AB}\cap\ol{PQ},\ol{CD}\cap\ol{PQ}$. 
\\[4pt]
Time for its more useful and/or famous counterpart...
\subsection{The dual}
\begin{thm}[Theorem \mdseries(Dual of \dots aka DDIT)]
Let quadrilateral $ABCD$ with inconic $\gamma$ have $E,F=\ol{AB}\cap\ol{CD},\ol{AD}\cap\ol{BC}$, and let $P$ be any point in the plane. Then there exists an involutive pairing on the pencil of lines through $P$, swapping $(\ol{PA},\ol{PC})$, $(\ol{PB},\ol{PD})$, $(\ol{PE},\ol{PF})$, and the (two) tangents from $P$ to $\gamma$.
\end{thm}
\begin{proof}Reciprocate (do pole/polar thing) wrt $\gamma$.\end{proof}
So how is this used? One of two things happens usually:
\begin{alphenum}
\item Used in a angle-reflective manner: isogonals, for instance;
\item Project the involutive pairs of lines onto some other line.
\end{alphenum}
Perhaps this is best seen by example.
\section{Standard examples}
We pull the two most famous examples from USAMO, as noted in lectures by Evan Chen and others.
\exercise[AMO 2012/5]Let $P$ be a point in the plane of $\triangle ABC$, and $\gamma$ a line through $P$. Let $A'$, $B'$, $C'$ be the points where the reflections of lines $PA, PB, PC$ with respect to $\gamma$ intersect lines $BC$, $CA$, $AB$ respectively. Prove that $A'$, $B'$, $C'$ are collinear. 
\exercise[AMO 2018/5]Let $ABCD$ be a convex cyclic quadrilateral with $E = \overline{AC} \cap \overline{BD}$, $F = \overline{AB} \cap \overline{CD}$, $G = \overline{DA} \cap \overline{BC}$. The circumcircle of $\triangle ABE$ intersects line $CB$ at $B$ and $P$, and the circumcircle of $\triangle ADE$ intersects line $CD$ at $D$ and $Q$. Assume $C$, $B$, $P$, $G$ and $C$, $Q$, $D$, $F$ are collinear in that order. Let $M = \overline{FP} \cap \overline{GQ}$. Prove that $\angle MAC = 90^\circ$.

\begin{remark}
I thought for so long that Desargues involution was somehow related to the more famous theorem (at least in old culture).\\
Even after hearing that phrase I didn't get it. I remember thinking, ``don't they say this problem is nuked by ...?'' Didn't make sense of the solutions probably because they were the official ones. Key missing word: \emph{dual}.
\end{remark}

\section{Pset}
An olympiad geometer's oblivion, especially after the agony of USAMO $2022$. Some of these are also standard. Sadly, it's very hard to make DDIT proposals\dots\\[4pt]
Also, this is nowhere near a complete list. Thanks to Eric Shen (CA) for teaching DDIT to me, and recommending some of the below problems.
\exercise[CAMO 2021/1]Let $ABC$ be an acute triangle, and let the feet of the altitudes from $A$, $B$, $C$ to $\overline{BC}$, $\overline{CA}$, $\overline{AB}$ be $D$, $E$, $F$, respectively. Points $X\ne F$ and $Y\ne E$ lie on lines $CF$ and $BE$ respectively such that $\angle XAD=\angle DAB$ and $\angle YAD=\angle DAC$. Prove that $X$, $D$, $Y$ are collinear.
\exercise[Shortlist 2007 G3]Let $ABCD$ be a trapezoid whose diagonals meet at $P$. Point $Q$ lies between parallel lines $BC$ and $AD$, and line $CD$ separates points $P$ and $Q$. Given that $\angle AQD = \angle CQB$, prove that $\angle BQP = \angle DAQ$. 
\exercise[Taiwan TST 2014/3/3] Let $M$ be any point on the circumcircle of triangle $ABC$. Suppose the tangents from $M$ to the incircle meet $\ol{BC}$ at two points $X_1$ and $X_2$.  Prove that the circumcircle of triangle $MX_1X_2$ intersects the circumcircle of $ABC$ again at the tangency point of the $A$-mixtilinear incircle.
\exercise[Serbia 2017/6] Let $ABC$ be a triangle. Suppose the two common tangents to its circumcircle and $A$-excircle meet line $BC$ at two points $P$ and $Q$. Prove that $\angle PAB = \angle CAQ$.
\exercise[USA TST 2018/5] Let $ABCD$ be a convex cyclic quadrilateral which is not a kite, but whose diagonals are perpendicular and meet at $H$. Denote by $M$ and $N$ the midpoints of $\overline{BC}$ and $\overline{CD}$. Rays $MH$ and $NH$ meet $\overline{AD}$ and $\overline{AB}$ at $S$ and $T$, respectively. Prove that there exists a point $E$, lying outside quadrilateral $ABCD$, such that
\begin{itemize}
\item ray $EH$ bisects both angles $\angle BES$, $\angle TED$, and
\item $\angle BEN = \angle MED$.
\end{itemize}.
\exercise[Shortlist 2012 G8] Let $ABC$ be a triangle with circumcircle $\omega$ and $\ell$ a line without common points with $\omega$. Denote by $P$ the foot of the perpendicular from the center of $\omega$ to $\ell$. The lines $BC$, $CA$, $AB$ intersect $\ell$ at the points $X$, $Y$, $Z$ different from $P$. Prove that the circumcircles of triangles $AXP$, $BYP$ and $CZP$ have a common point different from $P$ or are mutually tangent at $P$. 
\exercise[Shortlist 2021 G8 (sic), due to Tiger Zhang] Let $ABC$ be a triangle with circumcircle $\omega$ and let $\Omega_A$ be the $A$-excircle. Let $X$ and $Y$ be the intersection points of $\omega$ and $\Omega_A$. Let $P$ and $Q$ be the projections of $A$ onto the tangent lines to $\Omega_A$ at $X$ and $Y$ respectively. The tangent line at $P$ to the circumcircle of the triangle $APX$ intersects the tangent line at $Q$ to the circumcircle of the triangle $AQY$ at a point $R$. Prove that $\overline{AR} \perp \overline{BC}$.
\end{document}