\documentclass{seto}
\title{Monge}
\author{Neal Yan}
\date{MOSP 2025}
\begin{document}
\maketitle
\quote{{\small I've come to terms with Monge existing-
for there to be good theorems there also have to be bad ones,
and you need to accept them.}}{Aprameya Tripathy}
\begin{remark}[Credits] This lecture was originally given at 
G2 $2024$ and rewritten with much help from Patrick Suwanich. 
\end{remark}
\section{Motivation}
The reason for making this lecture is because 
I find Monge problems overrated in difficulty.\\
The most-used characteristic of the similicenter is that it lies on the line of centers. This allows 
one to connect otherwise unrelated lines. Plus, don't you think it's such a cool theorem? 
\section{Theory}
Depth limited, therefore this is just a list of fun facts apart from the first two results.
\begin{thm}[Fancy geometry parlance]
    Point $P$ is a \emph{similicenter} of two shapes $\mathcal A$, $\mathcal B$ if there's a homothety at 
    $P$ sending $\mathcal A\to\mathcal B$. 
    It is said to be an \emph{ex}similicenter or \emph{in}similicenter according to whether said homothety
    has positive or negative scale factor.
\end{thm}
Usually, the shapes in question are circles or polygons with $\le4$ sides.
\begin{thm}[Theorem (Monge)]
For any three circles $\omega_1,\omega_2,\omega_3$, pairwise exsimilicenters are collinear.
Also, the insimilicenter of two pairs of circles and 
the exsimilicenter of the third are collinear as well
concur.
\end{thm}
\begin{thm}[Lemma: characterise similicenter]
Let circles $\omega_1,\omega_2$ intersect at $A,B$. 
Point $X$ is a similicenter iff $XA=XB$ and 
\(\dangle AXB=\frac12\parenth{\widehat{AB}_{\omega_1}-\widehat{AB}_{\omega_2} }.\)
\end{thm}
\begin{thm}[Theorem (Pitot)]
Quadrilateral $ABCD$ has an incircle iff $AB+CD=AD+BC$, and has an excircle iff 
$AB+BC=AD+DC$ or $BA+AD=BC+CD$.
\end{thm}
\section{Problems}
Because Monge is either easy or hard to see, problems involving this
are always IMO$1$ or $\ge$IMO$2.5$ difficulty.
\exercise[Iran TST 2020] Let $ABC$ be an isosceles triangle with $AB=AC$ and incenter $I$. 
 Circle $\omega$ passes through $C$ and $I$ and is tangent to $AI$. 
 Circle $\omega$ intersects $AC$ and circumcircle of $ABC$ at $Q$ and $D$, respectively. 
 Let $M$ be the midpoint of $AB$ and $N$ be the midpoint of $CQ$. 
 Prove that $AD$, $MN$ and $BC$ are concurrent.
\footnote{Haruka Kimura found an absolutely brilliant radical axis solution
 so there are multiple nice ways to interpret the midpoints.}
\exercise[ARML 2024/I10] Circles $\omega_A,\omega_B,\gamma$ respectively 
have centres $A,B,O$ and radii $2$, $3$, $9$. $\omega_A$, $\omega_B$ are internally tangent to $\gamma$. The common external tangents of $\omega_A$ and $\omega_B$ meet at $T$. 
If $TO = 2AB$, what is $AB$?
\exercise[USA TST 2023/2] In acute triangle $ABC$, 
let $M$ be the midpoint of $BC$ and let $E$ and $F$ be the feet of the altitudes from $B$ and $C$, respectively. 
Let $K$ be the intersection of the common external tangents of $(BME)$ and $(CMF)$. 
Show that if $K\in (ABC)$,then $\ol{AK}\perp\ol{BC}$.
\exercise[EGMO 2016/4] Congruent circles $\omega_1$ and $\omega_2$ intersect at points $X_1$ and $X_2$. 
 Consider a circle $\omega$ externally tangent to $\omega_1$ at $T_1$ 
 and internally tangent to $\omega_2$ at point $T_2$. Prove that $\ol{X_1T_1}\cap\ol{X_2T_2}\in\omega$.
\exercise[ISL 2007/G8] Point $ P$ lies on side $AB$ of a convex quadrilateral 
 $ABCD$. Let $\omega$ be the incircle of triangle $CPD$, 
 and let $ I$ be its incenter. 
 Suppose that $\omega$ is tangent to the incircles of triangles $APD$ and 
 $BPC$ at points $K$ and $L$, respectively. 
 Let $E=\ol{AC}\cap\ol{BD}$, $F=\ol{AK}\cap\ol{BL}$.
 Prove that points $E$, $I$, and $F$ are collinear. 
\exercise[IMO 2008/6] Let $ABCD$ be a convex quadrilateral with $BA \neq BC$. 
Denote the incircles of triangles $ABC$, $ADC$ by $\omega_1$, $\omega_2$ respectively. 
Suppose that there exists a circle $\omega$ tangent to ray $BA$ beyond $A$ and $BC$ beyond $C$, 
as well as to lines $AD$ and $CD$.
Prove that the common external tangents to $\omega_1$ and $\omega_2$ intersect on $\omega$.
\exercise[ELMO SL 2024/G4, by me] In quadrilateral $ABCD$ with incenter $I$,
 points $W,X,Y,Z$ lie on sides $AB, BC,CD,DA$ with 
 $AZ=AW$, $BW=BX$, $CX=CY$, $DY=DZ$. 
 Define $T=\ol{AC}\cap\ol{BD}$ and $L=\ol{WY}\cap\ol{XZ}$. 
 Let points $O_a,O_b,O_c,O_d$ be such that 
 $\angle O_aZA=\angle O_aWA=90^\circ$ (and cyclic variants), 
 and $G=\ol{O_aO_c}\cap\ol{O_bO_d}$. Prove that $\ol{IL}\parallel\ol{TG}$.
\exercise[RMM 2010/3] Let $A_1A_2A_3A_4$ be a quadrilateral with no pair of parallel sides. 
For each $1\le i\le 4$, define $\omega_i$ as the circle tangent to the interior of $\ol{A_iA_{i+1}}$
and the extensions of $\ol{A_{i-1}A_i}$, $\ol{A_{i+1}A_{i+2}}$ (indices considered modulo $4$). 
Let $T_i=\omega_i\cap\ol{A_iA_{i+1}}$.
Prove that $\ol{A_1A_2}$, $\ol{A_3A_4}$, $\ol{T_2T_4}$ concur if and only if 
$\ol{A_2A_3}$, $\ol{A_4A_1}$, $\ol{T_1T_3}$ do.
\exercise[ISL 2017/G7] Quadrilateral $ABCD$ has incenter $I$. 
 Let $I_a, I_b, I_c$ and $I_d$ be the respective incenters of triangles $DAB, ABC, BCD$ and $CDA$, 
 Suppose that the common external tangents of $(AI_bI_d)$ and $(CI_bI_d)$ meet at $X$, 
 and those of the $(BI_aI_c)$ and $(DI_aI_c)$ meet at $Y$. 
 Prove that $\angle{XIY}=90^{\circ}$.
\exercise[ISL 2020/G5] Points $K, L, M, N$ are chosen on 
$\ol{AB}$, $\ol{BC}$, $\ol{CD}$, $\ol{DA}$ of cyclic quadrilateral so that
 $KLMN$ is a rhombus with $KL \parallel AC$, $LM \parallel BD$. 
 Let $\omega_A$, $\omega_B$, $\omega_C$, $\omega_D$ be the respective incircles of 
 triangles $ANK$, $BKL$, $CLM$, $DMN$. 
 Prove that the common internal tangents to $(\omega_A, \omega_C)$ and  
 $(\omega_B,\omega_D)$ are concurrent.
%\exercise[Serbia 2017/6] Let $k$ be the circumcircle of $\triangle ABC$ and let $k_a$ be the $A$-excircle.
%Let the two common tangents of $k,k_a$ intersect $BC$ at $P,Q$. 
%Prove that $\dangle PAB=\dangle CAQ$.
\exercise[ISL 2015/G7] (difficulty warning)
Points $P$, $Q$, $R$, $S$ are on sides $AB$, $BC$, $CD$, $DA$
of convex quadrilateral $ABCD$, respectively. Let $O=\ol{PR}\cap\ol{QS}$. 
Given that $APOS$, $BQOP$, $CROQ$, $DSOR$ each have an incircle,
show that $\ol{AC}$, $\ol{PQ}$, $\ol{RS}$ concur. 
\end{document}
